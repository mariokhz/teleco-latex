\documentclass{article}
\usepackage{xcolor}
\usepackage{graphicx}
\usepackage{tikz}
\usepackage{circuitikz}
\usepackage{pgfplots}
\usepackage{darkmode}
\usepackage{amsmath}
\usepackage[a4paper, top=2cm, bottom=2cm]{geometry}

\enabledarkmode

\definecolor{c1}{HTML}{46C2CB}
\definecolor{c2}{HTML}{f4895f}
\definecolor{c3}{HTML}{EB3678}
\definecolor{c4}{HTML}{FB773C}

\definecolor{page}{HTML}{262626}
\pagecolor{page}

\begin{document}
\title{Tareas Previas P2}
\author{Mario López Sáez}
\date{\today}
\maketitle

\hspace*{2.2cm} \begin{circuitikz}[american, cute inductors]
	\draw (0,0) to[open, o-] ++(0, 0) to[C] ++(2, 0) to[R=$R_1$] ++(0, 4) to[short, -*] ++(4, 0) coordinate(VCC);
	\draw (2, -4) to[R=$R_2$] ++(0, 4);
	\draw (4, 0) node[npn, anchor=B](Q1) {Q1} to[short] (2, 0);
	\draw (Q1.C) to[R=$R_C$] ++(0, 3.22);
	\draw (Q1.E) ++(0, -3.22) coordinate(tmp) to[R=$R_E$] (Q1.E);
	\draw (2, -4) to[short] (tmp);
	\coordinate (mid) at ($(2, -4)!.5!(tmp)$);
	\draw (mid) node[ground]{};
	\coordinate (aux) at (VCC |- Q1.C); % Esto toma la coordenada x de VCC y la y de Q1.C
	\draw (Q1.C) to[short, -*] (aux); % Dibuja algo desde esa coordenada
	\draw (aux) node[right] {$v_o$};
	\draw (VCC) node[right] {$+V_{CC}$};
	\draw (0, 0) node[left] {$v_i$};



\end{circuitikz}

\begin{center}
	Datos: $V_{CC} = 8 V$, $R_1 = 8.2 k\Omega$, $R_2 = 2.2 k\Omega$, $R_E = 1 k\Omega$, $R_C = 2.7 k\Omega$, $\beta = 300$, $V_{BE} = 0.6 V$
\end{center}


\newpage

\section{Cálculo de punto Q}
Calculamos el punto Q del circuito
\begin{flalign*}
    &V_{B} = V_{CC} \cdot \frac{R_{2}}{R_{1}+R_{2}} = 1.692 \, \text{V} \\
    &V_{E} = V_{B} - V_{BE} = 1.092 \, \text{V} \\
    &I_E = 1.092 \, \text{mA} \approx I_C \\
    &V_{C} = V_{CC} - I_{C} \cdot R_{C} = 5.05 \, \text{V} \\
    &V_{CE} = 3.96 \, \text{V} \\
    \\
\end{flalign*}

\section{Cálculo de la recta dinámica}


\begin{flushleft}
    \text{Recta de carga dinámica: } $i_C - I_{CQ} = m_d (v_{CE} - V_{CEQ})$ \\
    \text{Pendiente de la carga dinámica: } $m_d = - \frac{1}{R_{CA}}$
\end{flushleft}


\begin{circuitikz}[american]
    \coordinate (start) at (0, 0);
    \draw(start) to[short] (5, 0);
    \draw (5, 0) node[npn, anchor=B](Q1) {Q1};
    \draw (2, -3) node[ground](G1) {};
    \draw (2, 0) to[R=$R_1$] (G1);
    \draw (4, -3) node[ground](G2) {};
    \draw (4, 0) to[R=$R_2$] (G2);
    \draw (0, -3) node[ground](G3) {};
    \draw (start) to[vsourcesin,v=$v_{in}$] (G3);
    \draw (Q1.E) to[R=$R_E$] ++(0, -2.2) coordinate(RE);
    \draw (RE) node[ground]{};
    \draw (Q1.C) to[short] ++(2,0) coordinate(RC) to[short] ++(2,0) to[open, o-] ++(0,0) node[right] {$v_o$};
    \draw (RC) ++(0, -3.7) to[R=$R_C$] (RC);
    \draw (RC) ++(0, -3.7) node[ground]{};
\end{circuitikz}

\begin{flushleft}
	$m_d = - \frac{1}{R_C+R_E} = -0.270 \ \text{mA/V}$ \\ 
	$i_C - 1.092 = -0.27 (v_{CE} - 5.05)$ \\
	$v_{ce_{max}} = 2.455/0.27 = 9.0555 \ V$
\end{flushleft}

\begin{center}
	$i_C = 2.455 -0.27v_{CE}$

\end{center}
\begin{center}
	\begin{tikzpicture}
		\coordinate (start) at (0,0);
	
		\draw[->] (start) -- (0, 5.5) node[right] {$i_c (mA)$}; 
		\draw[->] (start) -- (6, 0) node[right] {$V_{CE}(V)$};
		\coordinate (icmax) at (0, 4.91);
		\coordinate (icq) at (0, 2.184);
		\draw[c2] (icmax) -- (9.0555/2, 0) node[below] {9.06};
		\coordinate (vceq) at (3.96/2, 0);
		\draw[c2] (icmax) node[left] {$ic_{max}=2.455$};
		\coordinate (mid) at (vceq |- icq);
		\fill[c2] (mid) circle (2pt) node[right] {$Q(3.96, 1.09)$};
		\coordinate (midx) at (vceq |- start);
		\coordinate (midy) at (start |- icq);

		\draw[dashed] (midx) -- (mid);
		\draw[dashed] (midy) -- (mid);
		\draw (icq) node[left] {1.09};
		\draw (vceq) node[below] {3.96};


\end{tikzpicture}

\end{center}

\hspace*{2.2cm} \begin{circuitikz}[american, cute inductors]
	\draw (0,0) to[open, o-] ++(0, 0) to[C] ++(2, 0) to[R=$R_1$] ++(0, 4) to[short, -*] ++(4, 0) coordinate(VCC);
	\draw (2, -4) to[R=$R_2$] ++(0, 4);
	\draw (4, 0) node[npn, anchor=B](Q1) {Q1} to[short] (2, 0);
	\draw (Q1.C) to[R=$R_C$] ++(0, 3.22);
	\draw (Q1.E) ++(0, -3.22) coordinate(tmp) to[R=$R_E$] (Q1.E);
	\draw (2, -4) to[short] (tmp);
	\draw[c1] (Q1.E) to[short] ++(1, 0) to[C] ++(0, -3) to[short] ++(-1, 0);
	\coordinate (mid) at ($(2, -4)!.5!(tmp)$);
	\draw (mid) node[ground]{};
	\coordinate (aux) at (VCC |- Q1.C); % Esto toma la coordenada x de VCC y la y de Q1.C
	\draw (Q1.C) to[short, -*] (aux); % Dibuja algo desde esa coordenada
	\draw (aux) node[right] {$v_o$};
	\draw (VCC) node[right] {$+V_{CC}$};
	\draw (0, 0) node[left] {$v_i$};



\end{circuitikz}

Lo analizamos en CA:

\begin{circuitikz}[american]
    \coordinate (start) at (0, 0);
    \draw(start) to[short] (5, 0);
    \draw (5, 0) node[npn, anchor=B](Q1) {Q1};
    \draw (2, -3) node[ground](G1) {};
    \draw (2, 0) to[R=$R_1$] (G1);
    \draw (4, -3) node[ground](G2) {};
    \draw (4, 0) to[R=$R_2$] (G2);
    \draw (0, -3) node[ground](G3) {};
    \draw (start) to[vsourcesin,v=$v_{in}$] (G3);
    \draw[c1] (Q1.E) to[short] ++(0, -2.2) coordinate(RE);
    \draw[c1] (RE) node[ground]{};
    \draw (Q1.C) to[short] ++(2,0) coordinate(RC) to[short] ++(2,0) to[open, o-] ++(0,0) node[right] {$v_o$};
    \draw (RC) ++(0, -3.7) to[R=$R_C$] (RC);
    \draw (RC) ++(0, -3.7) node[ground]{};
\end{circuitikz}

\begin{flushleft}
    $m_d = - \frac{1}{R_{C}} = 0.370$ \\
    $i_C - 1.09 = -0.37 (v_{CE} - 5.05)$ \\
    $v_{ce_{max}} = 2.468/0.37 = 6.6702 \ V$
\end{flushleft}

\begin{center}
	$i_C = 2.468 -0.37v_{CE}$

\end{center}

\begin{center}
	\begin{tikzpicture}
		\coordinate (start) at (0,0);
	
		\draw[->] (start) -- (0, 6.5) node[right] {$i_c (mA)$}; 
		\draw[->] (start) -- (6.5, 0) node[right] {$V_{CE}(V)$};
		\coordinate (icmax) at (0, 2.959*2);
		\coordinate (icq) at (0, 1.09*2);
		\draw[c1] (icmax) -- (10.96/2, 0) node[below] {6.67};
		\coordinate (vceq) at (5.05/2, 0);
		\draw (4, -0.2) node[] {8};
		\draw (icmax) node[left] {$ic_{max}=2.595$};
		\coordinate (mid) at (vceq |- icq);
		\fill[c2] (mid) circle (2pt) node[right] {$Q(6.67, 0.36)$};
		\coordinate (midx) at (vceq |- start);
		\coordinate (midy) at (start |- icq);

		\draw[dashed] (midx) -- (mid);
		\draw[dashed] (midy) -- (mid);
		\draw (icq) node[left] {0.36};

\end{tikzpicture}

\end{center}



\end{document}
